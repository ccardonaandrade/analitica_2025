\documentclass[letterpaper]{article}
\usepackage[utf8]{inputenc}
\usepackage[spanish]{babel}
\usepackage{amssymb}
\usepackage{tabularx}
\usepackage{amsthm}
\usepackage{amsmath}
\usepackage{graphicx}
\usepackage{epstopdf}
\usepackage{float}
\usepackage{amsfonts}
\usepackage{graphics}
\usepackage{graphpap}
\usepackage{latexsym}
\usepackage{makeidx}
\usepackage{enumerate}
\usepackage{rotating}
\usepackage{url}
\usepackage{multirow}
\usepackage{indentfirst}
%\usepackage[nofiglist,notablist,tablesfirst,nomarkers]{endfloat} % figuras al final
\usepackage[margin=3cm]{geometry}
\usepackage[colorlinks=true]{hyperref}

\usepackage{multicol} % Paquete necesario para las columnas


	

\date{\vspace{-5ex}}
\begin{document}

\title{{\bf Pre-Parcial}} 
\author{Analítica de los Negocios \\
	Pontificia Universidad Javeriana}
\date{}

\maketitle 


El conjunto de datos para este ejercicio proviene de la base de datos \hyperlink{https://github.com/jldbc/coffee-quality-database}{Coffee Quality} y fue obtenido del repositorio de GitHub de \hyperlink{https://github.com/rfordatascience/tidytuesday/blob/master/data/2020/2020-07-07/readme.md}{TidyTuesday}. Contiene datos detallados de más de 1000 variedades de café, incluyendo su procedencia, productor, características específicas y evaluación de calidad.

\medskip

Realicen los siguientes ejercicios en orden, siguiendo cada paso cuidadosamente. Incluyan nombres en los ejes y un título informativo para todos los gráficos. Escriban todas las interpretaciones en el contexto de los datos.

\section{Ejercicios}

\begin{enumerate}

\item Carguen los datos `coffee\_ratings.csv' a R.

\item Eliminen las observaciones para las cuales total\_cup\_points==0



\item Estimen la regresión lineal:

 $\widehat{\text{Total Cup Points}}=\hat{\beta_0}+\hat{\beta_1}\times Aroma$
 
 Interpreten la pendiente en el contexto de los datos.
 
 \item ¿Tiene sentido interpretar el intercepto? Si es así, escriban la interpretación en el contexto de los datos.
 
 \item ¿Tomarían una taza de café representada por el intercepto? Grafiquen la densidad de la variable dependiente y una línea punteada señalando la ubicación del intercepto en la gráfica.
 
 \item Ahora evaluemos las condiciones del modelo. Verifiquen las condiciones de linealidad, varianza constante y normalidad. Para cada condición, indiquen si se cumple junto con una breve explicación de su conclusión. Incluyan los gráficos y/o estadísticas descriptivas que utilizaron para justificar su respuesta.
 
 
 \item Ahora realicen la prueba de hipótesis para la pendiente. En su respuesta, enuncien las hipótesis nula y alternativa en palabras, y expongan la conclusión en el contexto de los datos.
 
 \item Interpreten el $R^2$
 
 \item Creen una variable dummy llamada $colombia$ que sea igual a 1 si el país de origen es Colombia. ¿Qué porcentaje de los cafés en la muestra provienen de Colombia?
 
 \item Estimen la regresión lineal:
 
 
 $\widehat{\text{Total Cup Points}}=\hat{\beta_0}+\hat{\beta_1}\times Aroma + \hat{\beta_2}\times Colombia$
 
 Interpreten $\hat{\beta_2}$ en el contexto de los datos.
 
 \item ¿Cómo cambia el $R^2$ con respecto al de la regresión en el punto 6?¿Qué nos dice esto sobre la variable $colombia$?
 
 
 \item Estimen la regresión lineal:
 
 $\widehat{\text{Total Cup Points}}=\hat{\beta_0}+\hat{\beta_1}\times Flavor$
 
 Interpreten el $R^2$. Al comparar con el punto 11, ¿qué pueden decir sobre las variables $aroma$ y $flavour$ y su relación con total\_cup\_points?
 
 \item Imaginen que tienen que hablar con un caficultor que está en el dilema entre mejorar el aroma o el sabor del café que produce. Basado en estos datos y en la regresión:
 
  $\widehat{\text{Total Cup Points}}=\hat{\beta_0}+\hat{\beta_1}\times Aroma + \hat{\beta_2}\times Flavor$
  
  ¿Qué le recomendarían?
  
 \item Elijan una de las características del punto 6 y estimen la regresión:
  
  $\widehat{\text{log(Total Cup Points)}}=\hat{\beta_0}+\hat{\beta_1}\times \text{log(Característica)}$
  
  Interpreten la pendiente en el contexto de los datos. 
  
  
\end{enumerate}


\end{document}